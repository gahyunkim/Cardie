%%%%%%%%%%%%%%%%%%%%%%%%%%%%%%%%%%%%%%%%%
% Structured General Purpose Assignment
% LaTeX Template
%
% This template has been downloaded from:
% http://www.latextemplates.com
%
% Original author:
% Ted Pavlic (http://www.tedpavlic.com)
%
% Note:
% The \lipsum[#] commands throughout this template generate dummy text
% to fill the template out. These commands should all be removed when 
% writing assignment content.
%
%%%%%%%%%%%%%%%%%%%%%%%%%%%%%%%%%%%%%%%%%

%----------------------------------------------------------------------------------------
%	PACKAGES AND OTHER DOCUMENT CONFIGURATIONS
%----------------------------------------------------------------------------------------

\documentclass{article}
\usepackage[numbered, framed]{matlab-prettifier}
\usepackage{amsmath}%
\usepackage{MnSymbol}%
\usepackage{wasysym}%
\usepackage{cancel}
\usepackage{tikz}
\usepackage{graphicx}
\graphicspath{ {C:/Users/Roman/Documents/CS326/team-project-client-template/ERDiagram} }
\usepackage[margin=1in]{geometry} 
\usetikzlibrary{automata,positioning}
\usepackage{fancyhdr} % Required for custom headers
\usepackage{lastpage} % Required to determine the last page for the footer
\usepackage{extramarks} % Required for headers and footers
\usepackage{graphicx} % Required to insert images
\usepackage{lipsum} % Used for inserting dummy 'Lorem ipsum' text into the template
\usepackage{enumitem}
% Margins
\topmargin=-0.45in
\evensidemargin=0in
\oddsidemargin=0in
\textwidth=6.5in
\textheight=9.0in
\headsep=0.25in 

\linespread{1.1} % Line spacing

% Set up the header and footer
\pagestyle{fancy}
\lhead{\hmwkAuthorName} % Top left header
\chead{\hmwkClass\ (\hmwkClassInstructor\ \hmwkClassTime): \hmwkTitle} % Top center header
\rhead{\firstxmark} % Top right header
\lfoot{\lastxmark} % Bottom left footer
\cfoot{} % Bottom center footer
\rfoot{Page\ \thepage\ of\ \pageref{LastPage}} % Bottom right footer
\renewcommand\headrulewidth{0.4pt} % Size of the header rule
\renewcommand\footrulewidth{0.4pt} % Size of the footer rule

\setlength\parindent{0pt} % Removes all indentation from paragraphs

%----------------------------------------------------------------------------------------
%	DOCUMENT STRUCTURE COMMANDS
%	Skip this unless you know what you're doing
%----------------------------------------------------------------------------------------

% Header and footer for when a page split occurs within a problem environment
\newcommand{\enterProblemHeader}[1]{
\nobreak\extramarks{#1}{#1 continued on next page\ldots}\nobreak
\nobreak\extramarks{#1 (continued)}{#1 continued on next page\ldots}\nobreak
}

% Header and footer for when a page split occurs between problem environments
\newcommand{\exitProblemHeader}[1]{
\nobreak\extramarks{#1 (continued)}{#1 continued on next page\ldots}\nobreak
\nobreak\extramarks{#1}{}\nobreak
}

\setcounter{secnumdepth}{0} % Removes default section numbers
\newcounter{homeworkProblemCounter} % Creates a counter to keep track of the number of problems

\newcommand{\homeworkProblemName}{}
\newenvironment{homeworkProblem}[1][Problem \arabic{homeworkProblemCounter}]{ % Makes a new environment called homeworkProblem which takes 1 argument (custom name) but the default is "Problem #"
\stepcounter{homeworkProblemCounter} % Increase counter for number of problems
\renewcommand{\homeworkProblemName}{#1} % Assign \homeworkProblemName the name of the problem
\section{\homeworkProblemName} % Make a section in the document with the custom problem count
\enterProblemHeader{} % Header and footer within the environment
}{
\exitProblemHeader{} % Header and footer after the environment
}

\newcommand{\problemAnswer}[1]{ % Defines the problem answer command with the content as the only argument
\noindent\framebox[\columnwidth][c]{\begin{minipage}{0.98\columnwidth}#1\end{minipage}} % Makes the box around the problem answer and puts the content inside
}

\newcommand{\homeworkSectionName}{}
\newenvironment{homeworkSection}[1]{ % New environment for sections within homework problems, takes 1 argument - the name of the section
%\renewcommand{\homeworkSectionName}{#1} % Assign \homeworkSectionName to the name of the section from the environment argument
\subsection{\homeworkSectionName} % Make a subsection with the custom name of the subsection
\enterProblemHeader{ [\homeworkSectionName]} % Header and footer within the environment
}{
\enterProblemHeader{} % Header and footer after the environment
}
   
%----------------------------------------------------------------------------------------
%	NAME AND CLASS SECTION
%----------------------------------------------------------------------------------------

\newcommand{\hmwkTitle}{HTTP Request/Server} % Assignment title
\newcommand{\hmwkDueDate}{Thursday,\ December 1,\ 2016} % Due date
\newcommand{\hmwkClass}{CS\ 326} % Course/class
\newcommand{\hmwkClassTime}{} % Class/lecture time
\newcommand{\hmwkClassInstructor}{Professor Tim Richards} % Teacher/lecturer
\newcommand{\hmwkAuthorName}{} % Your name

%----------------------------------------------------------------------------------------
%	TITLE PAGE
%----------------------------------------------------------------------------------------

\title{
\vspace{2in}
\textmd{\textbf{\hmwkClass:\ \hmwkTitle}}\\
\normalsize\vspace{0.1in}\small{Due\ on\ \hmwkDueDate}\\
\vspace{0.1in}\large{ \hmwkClassInstructor}
\vspace{3in}
}
\author{Daanial Ahmed, Richard Cui, Roman Ganchin, \\Gahyun (Susie) Kim, Greg McGrath, Francis Phan}
\date{} % Insert date here if you want it to appear below your name

%----------------------------------------------------------------------------------------

\begin{document}

\maketitle


%----------------------------------------------------------------------------------------
%	TABLE OF CONTENTS
%----------------------------------------------------------------------------------------

%\setcounter{tocdepth}{1} % Uncomment this line if you don't want subsections listed in the ToC

%\newpage
%\tableofcontents
\newpage

%----------------------------------------------------------------------------------------
%	PROBLEM 9.10.1
%----------------------------------------------------------------------------------------
\begin{homeworkProblem}[HTTP Requests]
\begin{lstlisting}
GET /items/:itemid
\end{lstlisting} 
Returns the item with :itemid.\\
\begin{lstlisting}
GET /profile/:userid
\end{lstlisting} 
Returns the profile for user with :userid.\\
\begin{lstlisting}
GET /user/:userid/feed
\end{lstlisting} 
\indent Returns the feed of the user with id :userid.\\
\begin{lstlisting}
POST /upload/:userid [item]
\end{lstlisting} 
Where [item] is a JSON object describing a new item for sale where :userid is the vendor of that item.\\
\begin{lstlisting}
PUT/users/:userid/feeds/items/:itemid/like
\end{lstlisting} 
Current user has their ID pushed on to the item's likeCounter with id :itemid.\\
\begin{lstlisting}
PUT /users/:userid/feeds/items/:itemid/dislike
\end{lstlisting} 
Current user has their ID pushed on to the item's dislikeCounter with id :itemid.\\
\begin{lstlisting}
PUT /users/:userid/messages
\end{lstlisting} 
Sends a message which gets logged by both the sender and receiver\\
\begin{lstlisting}
GET /user/:userid/pm
\end{lstlisting} 
Returns the productManager of a user with :userid with all their items for sale.\\
\begin{lstlisting}
GET /users/:userid/messages
\end{lstlisting} 
Returns a list of all the messages for a particular user\\
\begin{lstlisting}
GET /profile/:userid
\end{lstlisting} 
Gets the current user's profile\\
\begin{lstlisting}
DELETE /user/:userid/pm/item/:itemid
\end{lstlisting} 
Item with :itemid is deleted from user with :userid productManager as it has been sold or removed by user. Also removes the item from all other feeds as it is no longer for sale.\\
\begin{lstlisting}
POST /resetdb
\end{lstlisting} 
Resets database on application.\\

\end{homeworkProblem}

% To have just one problem per page, simply put a \clearpage after each problem
\begin{homeworkProblem}[Contributions]

\begin{homeworkSection}{Daanial Ahmed}
Daanial Ahmed contributed to the Upload component.
\end{homeworkSection}

\begin{homeworkSection}{Richard Cui}
Richard Cui contributed to the Chat component server methods and schemas.
\end{homeworkSection}

\begin{homeworkSection}{Roman Ganchin}
Roman Ganchin contributed to the Trending, Category, and Item components.
\end{homeworkSection}

\begin{homeworkSection}{Gahyun (Susie) Kim}
Gahyun (Susie) Kim contributed to the User component.
\end{homeworkSection}

\begin{homeworkSection}{Greg McGrath}
Greg McGrath contributed to the Feed and item components. 
\end{homeworkSection}

\begin{homeworkSection}{Francis Phan}
Francis Phan contributed to the ProductManager and Item components, the dummy database, and linking and routing all of the components into one application.
\end{homeworkSection}

\end{homeworkProblem}

%----------------------------------------------------------------------------------------
%	PROBLEM 5
%----------------------------------------------------------------------------------------

\begin{homeworkProblem}[Cut Features]
No features were cut from the project, though some features were pushed back until a server/database is implemented as they are not currently feasible at this time.
\end{homeworkProblem}

\end{document}
\begin{homeworkProblem}[Diagram]
 \includegraphics[scale=.65]{ERDiagram.png}
\end{homeworkProblem}
\clearpage
\begin{homeworkProblem}[Entity Description]
\textbf{USER}: Anyone with an account in the system, that is formally logged in. A USER must have a \emph{name}, valid \emph{email}, and \emph{location}.
\newline

\textbf{FEED}: The FEED lists \emph{ITEMS} publicly open for sale by \textit{USERS}.
\newline

\textbf{ITEM}: A unique ITEM in our database. An item only ever belongs to a single \emph{USER}. A \emph{USER} can have many ITEMS for sale. An ITEM must have a \emph{photo}, \emph{name}, \emph{description}. 
\newline

\textbf{PRODUCT MANAGER}: A interface in which a USER can see all \textit{ITEMS} they have open for sale. Every \emph{USER} has one and only one.
\newline

\textbf{CHAT}: A CHAT must have exactly 2 \emph{users} associated with it. A \emph{USER} can have multiple CHATS with different people. 
\newline

\textbf{SETTINGS}: The set of preferences a \emph{USER} has. Ex. "Item search radius=25 miles" 
\newline

\textbf{VIEWER}: Someone not formally logged within our system. They have a \emph{FEED}, and can look at \emph{items}.\newline
\end{homeworkProblem}
\begin{homeworkProblem}[Page/Widget Description]
\textbf{Main UI(public and private)}: the Main UI displays the \emph{Feed} which consists of \emph{Items}. The \emph{Feed} can be utilized by either a \emph{Viewer} or a \emph{User}. \newline

\textbf{Trending Items}: the Trending Items page displays an assortment of popular \emph{Item$'$s} based on category.
\newline
\textbf{Sell Item}: this page allows a \emph{User} to upload an \emph{Item}. \newline

\textbf{User Profile}: the User Profile displays the components of the \emph{User}. It also contains its \emph{Settings} and \emph{Product Manager}. \newline

\textbf{Chat}: the Chat displays all \emph{Chat} entities belonging to a \emph{User}. \newline
 
\textbf{Item Manager}: the Item Manager displays the \emph{Product Manager} entity which displays all \emph{Item$'$s} put out by the \emph{User}. \newline
\end{homeworkProblem}
 
%----------------------------------------------------------------------------------------

\end{document}
