%%%%%%%%%%%%%%%%%%%%%%%%%%%%%%%%%%%%%%%%%
% Structured General Purpose Assignment
% LaTeX Template
%
% This template has been downloaded from:
% http://www.latextemplates.com
%
% Original author:
% Ted Pavlic (http://www.tedpavlic.com)
%
% Note:
% The \lipsum[#] commands throughout this template generate dummy text
% to fill the template out. These commands should all be removed when 
% writing assignment content.
%
%%%%%%%%%%%%%%%%%%%%%%%%%%%%%%%%%%%%%%%%%

%----------------------------------------------------------------------------------------
%	PACKAGES AND OTHER DOCUMENT CONFIGURATIONS
%----------------------------------------------------------------------------------------

\documentclass{article}
\usepackage[numbered, framed]{matlab-prettifier}
\usepackage{amsmath}%
\usepackage{MnSymbol}%
\usepackage{wasysym}%
\usepackage{cancel}
\usepackage{tikz}
\usepackage{graphicx}
\graphicspath{ {C:/Users/Roman/Documents/CS326/team-project-client-template/ERDiagram} }
\usepackage[margin=1in]{geometry} 
\usetikzlibrary{automata,positioning}
\usepackage{fancyhdr} % Required for custom headers
\usepackage{lastpage} % Required to determine the last page for the footer
\usepackage{extramarks} % Required for headers and footers
\usepackage{graphicx} % Required to insert images
\usepackage{lipsum} % Used for inserting dummy 'Lorem ipsum' text into the template
\usepackage{enumitem}
% Margins
\topmargin=-0.45in
\evensidemargin=0in
\oddsidemargin=0in
\textwidth=6.5in
\textheight=9.0in
\headsep=0.25in 

\linespread{1.1} % Line spacing

% Set up the header and footer
\pagestyle{fancy}
\lhead{\hmwkAuthorName} % Top left header
\chead{\hmwkClass\ (\hmwkClassInstructor\ \hmwkClassTime): \hmwkTitle} % Top center header
\rhead{\firstxmark} % Top right header
\lfoot{\lastxmark} % Bottom left footer
\cfoot{} % Bottom center footer
\rfoot{Page\ \thepage\ of\ \pageref{LastPage}} % Bottom right footer
\renewcommand\headrulewidth{0.4pt} % Size of the header rule
\renewcommand\footrulewidth{0.4pt} % Size of the footer rule

\setlength\parindent{0pt} % Removes all indentation from paragraphs

%----------------------------------------------------------------------------------------
%	DOCUMENT STRUCTURE COMMANDS
%	Skip this unless you know what you're doing
%----------------------------------------------------------------------------------------

% Header and footer for when a page split occurs within a problem environment
\newcommand{\enterProblemHeader}[1]{
\nobreak\extramarks{#1}{#1 continued on next page\ldots}\nobreak
\nobreak\extramarks{#1 (continued)}{#1 continued on next page\ldots}\nobreak
}

% Header and footer for when a page split occurs between problem environments
\newcommand{\exitProblemHeader}[1]{
\nobreak\extramarks{#1 (continued)}{#1 continued on next page\ldots}\nobreak
\nobreak\extramarks{#1}{}\nobreak
}

\setcounter{secnumdepth}{0} % Removes default section numbers
\newcounter{homeworkProblemCounter} % Creates a counter to keep track of the number of problems

\newcommand{\homeworkProblemName}{}
\newenvironment{homeworkProblem}[1][Problem \arabic{homeworkProblemCounter}]{ % Makes a new environment called homeworkProblem which takes 1 argument (custom name) but the default is "Problem #"
\stepcounter{homeworkProblemCounter} % Increase counter for number of problems
\renewcommand{\homeworkProblemName}{#1} % Assign \homeworkProblemName the name of the problem
\section{\homeworkProblemName} % Make a section in the document with the custom problem count
\enterProblemHeader{} % Header and footer within the environment
}{
\exitProblemHeader{} % Header and footer after the environment
}

\newcommand{\problemAnswer}[1]{ % Defines the problem answer command with the content as the only argument
\noindent\framebox[\columnwidth][c]{\begin{minipage}{0.98\columnwidth}#1\end{minipage}} % Makes the box around the problem answer and puts the content inside
}

\newcommand{\homeworkSectionName}{}
\newenvironment{homeworkSection}[1]{ % New environment for sections within homework problems, takes 1 argument - the name of the section
%\renewcommand{\homeworkSectionName}{#1} % Assign \homeworkSectionName to the name of the section from the environment argument
\subsection{\homeworkSectionName} % Make a subsection with the custom name of the subsection
\enterProblemHeader{ [\homeworkSectionName]} % Header and footer within the environment
}{
\enterProblemHeader{} % Header and footer after the environment
}
   
%----------------------------------------------------------------------------------------
%	NAME AND CLASS SECTION
%----------------------------------------------------------------------------------------

\newcommand{\hmwkTitle}{Adding a Database} % Assignment title
\newcommand{\hmwkDueDate}{Wednesday,\ December 14,\ 2016} % Due date
\newcommand{\hmwkClass}{CS\ 326} % Course/class
\newcommand{\hmwkClassTime}{} % Class/lecture time
\newcommand{\hmwkClassInstructor}{Professor Tim Richards} % Teacher/lecturer
\newcommand{\hmwkAuthorName}{} % Your name

%----------------------------------------------------------------------------------------
%	TITLE PAGE
%----------------------------------------------------------------------------------------

\title{
\vspace{2in}
\textmd{\textbf{\hmwkClass:\ \hmwkTitle}}\\
\normalsize\vspace{0.1in}\small{Due\ on\ \hmwkDueDate}\\
\vspace{0.1in}\large{ \hmwkClassInstructor}
\vspace{3in}
}
\author{Daanial Ahmed, Richard Cui, Roman Ganchin, \\Gahyun (Susie) Kim, Greg McGrath, Francis Phan}
\date{} % Insert date here if you want it to appear below your name

%----------------------------------------------------------------------------------------

\begin{document}

\maketitle


%----------------------------------------------------------------------------------------
%	TABLE OF CONTENTS
%----------------------------------------------------------------------------------------

%\setcounter{tocdepth}{1} % Uncomment this line if you don't want subsections listed in the ToC

%\newpage
%\tableofcontents
\newpage

%----------------------------------------------------------------------------------------
%	PROBLEM 9.10.1
%----------------------------------------------------------------------------------------
\newpage
\begin{homeworkProblem}[Special Server Setup Procedure]
We do not have any advanced features that require additional setup.
\end{homeworkProblem}
% To have just one problem per page, simply put a \clearpage after each problem
\begin{homeworkProblem}[Individual Contributions]
\renewcommand\labelitemi{$\textendash$}
\begin{enumerate}
\item \textbf{Daanial Ahmed}
 \begin{itemize}
  \item One entry in the list
  \item Another entry in the list
\end{itemize}
\item \textbf{Richard Cui} \begin{itemize}
  \item One entry in the list
  \item Another entry in the list
\end{itemize}
\item \textbf{Roman Ganchin} \begin{itemize}
  \item Changed getCategories to work with MongoDB
  \item Changed getCategorySync to work with MongoDB
  \item app.get('/users/:userid/feeds/categories') changed http GET request for categories. 
  \item Stringified trending.js in client
  \item Made resolveUserObjects for likeItem and dislikeItem to get them to work with MongoDB
  \item Implemented dislikeItem to work with MongoDB
\end{itemize}
\item \textbf{Gahyun (Susie) Kim} \begin{itemize}
  \item Added MongoDB and Express to server node\_modules
  \item Changed mock database IDs to MongoDB ObjectIDs and changed Integer type to String in schema, changed token
  \item Stringified app.js, feed, chat, item, sendchat, and upload 
  \item Created function getUserProfile to retrieve user profile data with the given ID
  \item Updated GET/profile/:userid
\end{itemize}
\item \textbf{Greg McGrath} \begin{itemize}
  \item Fixed getUserIdFromToken to work with MongoDB
  \item Changed getFeedData to work with MongoDB
  \item Fixed app.get(/user/:userid/feed) for feed
  \item Changed getItem to work with MongoDB
  \item Changed app.get('/items/:itemid) for get item
  \item Implemented likeItem to work with MongoDB
\end{itemize} 
\item \textbf{Francis Phan} \begin{itemize}
  \item GET Product Manager now fully interacts with MongoDB
  \item getProductManager added to accommodate mongoDB
\end{itemize} 
\end{enumerate}

\end{homeworkProblem}

%----------------------------------------------------------------------------------------
%	PROBLEM 5
%----------------------------------------------------------------------------------------

\begin{homeworkProblem}[Lingering Bugs/Issues/Dropped Features]

\textbf{Bugs/Issues}
\begin{itemize}
\item What are the bus and issues that we have?
\end{itemize}

\end{homeworkProblem}
\newpage
%----------------------------------------------------------------------------------------

\end{document}
