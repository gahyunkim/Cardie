%%%%%%%%%%%%%%%%%%%%%%%%%%%%%%%%%%%%%%%%%
% Structured General Purpose Assignment
% LaTeX Template
%
% This template has been downloaded from:
% http://www.latextemplates.com
%
% Original author:
% Ted Pavlic (http://www.tedpavlic.com)
%
% Note:
% The \lipsum[#] commands throughout this template generate dummy text
% to fill the template out. These commands should all be removed when 
% writing assignment content.
%
%%%%%%%%%%%%%%%%%%%%%%%%%%%%%%%%%%%%%%%%%

%----------------------------------------------------------------------------------------
%	PACKAGES AND OTHER DOCUMENT CONFIGURATIONS
%----------------------------------------------------------------------------------------

\documentclass{article}

\usepackage{fancyhdr} % Required for custom headers
\usepackage{lastpage} % Required to determine the last page for the footer
\usepackage{extramarks} % Required for headers and footers
\usepackage{graphicx} % Required to insert images
\usepackage{lipsum} % Used for inserting dummy 'Lorem ipsum' text into the template

% Margins
\topmargin=-0.45in
\evensidemargin=0in
\oddsidemargin=0in
\textwidth=6.5in
\textheight=9.0in
\headsep=0.25in 

\linespread{1.1} % Line spacing

% Set up the header and footer
\pagestyle{fancy}
\lhead{\hmwkAuthorName} % Top left header
\chead{\hmwkClass\ (\hmwkClassInstructor\ \hmwkClassTime): \hmwkTitle} % Top center header
\rhead{\firstxmark} % Top right header
\lfoot{\lastxmark} % Bottom left footer
\cfoot{} % Bottom center footer
\rfoot{Page\ \thepage\ of\ \pageref{LastPage}} % Bottom right footer
\renewcommand\headrulewidth{0.4pt} % Size of the header rule
\renewcommand\footrulewidth{0.4pt} % Size of the footer rule

\setlength\parindent{0pt} % Removes all indentation from paragraphs

%----------------------------------------------------------------------------------------
%	DOCUMENT STRUCTURE COMMANDS
%	Skip this unless you know what you're doing
%----------------------------------------------------------------------------------------

% Header and footer for when a page split occurs within a problem environment
\newcommand{\enterProblemHeader}[1]{
\nobreak\extramarks{#1}{#1 continued on next page\ldots}\nobreak
\nobreak\extramarks{#1 (continued)}{#1 continued on next page\ldots}\nobreak
}

% Header and footer for when a page split occurs between problem environments
\newcommand{\exitProblemHeader}[1]{
\nobreak\extramarks{#1 (continued)}{#1 continued on next page\ldots}\nobreak
\nobreak\extramarks{#1}{}\nobreak
}

\setcounter{secnumdepth}{0} % Removes default section numbers
\newcounter{homeworkProblemCounter} % Creates a counter to keep track of the number of problems

\newcommand{\homeworkProblemName}{}
\newenvironment{homeworkProblem}[1][Problem \arabic{homeworkProblemCounter}]{ % Makes a new environment called homeworkProblem which takes 1 argument (custom name) but the default is "Problem #"
\stepcounter{homeworkProblemCounter} % Increase counter for number of problems
\renewcommand{\homeworkProblemName}{#1} % Assign \homeworkProblemName the name of the problem
\section{\homeworkProblemName} % Make a section in the document with the custom problem count
\enterProblemHeader{\homeworkProblemName} % Header and footer within the environment
}{
\exitProblemHeader{\homeworkProblemName} % Header and footer after the environment
}

\newcommand{\problemAnswer}[1]{ % Defines the problem answer command with the content as the only argument
\noindent\framebox[\columnwidth][c]{\begin{minipage}{0.98\columnwidth}#1\end{minipage}} % Makes the box around the problem answer and puts the content inside
}

\newcommand{\homeworkSectionName}{}
\newenvironment{homeworkSection}[1]{ % New environment for sections within homework problems, takes 1 argument - the name of the section
\renewcommand{\homeworkSectionName}{#1} % Assign \homeworkSectionName to the name of the section from the environment argument
\subsection{\homeworkSectionName} % Make a subsection with the custom name of the subsection
\enterProblemHeader{\homeworkProblemName\ [\homeworkSectionName]} % Header and footer within the environment
}{
\enterProblemHeader{\homeworkProblemName} % Header and footer after the environment
}
   
%----------------------------------------------------------------------------------------
%	NAME AND CLASS SECTION
%----------------------------------------------------------------------------------------

\newcommand{\hmwkTitle}{Report\ \#5} % Assignment title
\newcommand{\hmwkDueDate}{Tuesday,\ November\ 15,\ 2016} % Due date
\newcommand{\hmwkClass}{CS\ 326} % Course/class
\newcommand{\hmwkClassTime}{} % Class/lecture time
\newcommand{\hmwkClassInstructor}{RedWine} % Teacher/lecturer
\newcommand{\hmwkAuthorName}{} % Your name

%----------------------------------------------------------------------------------------
%	TITLE PAGE
%----------------------------------------------------------------------------------------

\title{
\vspace{2in}
\textmd{\textbf{\hmwkClass:\ \hmwkTitle}}\\
\normalsize\vspace{0.1in}\small{Due\ on\ \hmwkDueDate}\\
\vspace{0.1in}\large{\textit{\hmwkClassInstructor\ \hmwkClassTime}}
\vspace{3in}
}

\author{Daanial Ahmed, Richard Cui, Roman Ganchin, \\Gahyun (Susie) Kim, Greg McGrath, Francis Phan}
\date{} % Insert date here if you want it to appear below your name

%----------------------------------------------------------------------------------------

\begin{document}

\maketitle

%----------------------------------------------------------------------------------------
%	TABLE OF CONTENTS
%----------------------------------------------------------------------------------------

%\setcounter{tocdepth}{1} % Uncomment this line if you don't want subsections listed in the ToC

%\newpage
%\tableofcontents
\newpage

%----------------------------------------------------------------------------------------
%	PROBLEM 1
%----------------------------------------------------------------------------------------

% To have just one problem per page, simply put a \clearpage after each problem

\begin{homeworkProblem}[Updated ER Diagram]
\includegraphics[scale=.65]{img/ERDiagramNew.png}
\end{homeworkProblem}

%----------------------------------------------------------------------------------------
%	PROBLEM 2
%----------------------------------------------------------------------------------------

\begin{homeworkProblem}[Outstanding Bugs] % Custom section title
There are a number of outstanding bugs in the current iteration of our web application. Most of these issues are caused by the lack of implementation of a database. We currently have pages linked together, and are able to pull dummy data from a JSON file, though the lack of a real database prevents us from being able to add new entries or modify existing entries, producing a static UI that can't take inputs. Additionally, there is currently an issue with parsing user information from JSON so we are hard-coding user ids.\\

As of right now, we intend to use Socket.io and Express to implement a chat client for users. Unfortunately, the chat client is not ready yet because we had some configuration issues, so the chat client has been tabled for later implementation. \\

Some UI elements are currently unpolished because we have not yet decided on a unified scheme for the UI, and some assets are missing and instead have placeholders.
\end{homeworkProblem}

%----------------------------------------------------------------------------------------
%	PROBLEM 3
%----------------------------------------------------------------------------------------

\begin{homeworkProblem}[Key React Components]

%--------------------------------------------

\begin{homeworkSection}{Feed} % Using the problem name elsewhere
The Feed component contains the feed of item images that is tailored to each user. Users can like or dislike items on the feed to see new items.
\end{homeworkSection}

\begin{homeworkSection}{Trending}
The Trending component draws items from the database and displays them to the user, sorted by categories. Unlike the Feed component, which can only display one item at a time, the Trending component can display multiple items at a time.
\end{homeworkSection}

\begin{homeworkSection}{Category}
The Category component is built into the Trending component and renders all items of a specific category.
\end{homeworkSection}

\begin{homeworkSection}{Profile}
The Profile component displays user settings. Users can edit their personal information through this component, as well as access their ProductManager and Chat.
\end{homeworkSection}

\begin{homeworkSection}{ProductManager}
The ProductManager component allows the user to manage their items for sale by editing or removing items through a number of fields.
\end{homeworkSection}

\begin{homeworkSection}{Chat}
The Chat component is a simple chat interface where two users can exchange messages.
\end{homeworkSection}

\begin{homeworkSection}{Upload}
The Upload component is a form submission that allows the user to upload items for sale into the application database.
\end{homeworkSection}

\begin{homeworkSection}{Item}
The Item component contains information specific to a single item, containing both item information as well as user iteractions with the item, such as likes/dislikes.
\end{homeworkSection}

\end{homeworkProblem}

%----------------------------------------------------------------------------------------
%	PROBLEM 4
%----------------------------------------------------------------------------------------

\begin{homeworkProblem}[Contributions]

\begin{homeworkSection}{Daanial Ahmed}
Daanial Ahmed contributed to the Upload component.
\end{homeworkSection}

\begin{homeworkSection}{Richard Cui}
Richard Cui contributed to the Chat component and the written report.
\end{homeworkSection}

\begin{homeworkSection}{Roman Ganchin}
Roman Ganchin contributed to the Trending, Category, and Item components.
\end{homeworkSection}

\begin{homeworkSection}{Gahyun (Susie) Kim}
Gahyun (Susie) Kim contributed to the User component.
\end{homeworkSection}

\begin{homeworkSection}{Greg McGrath}
Greg McGrath contributed to the Feed and item components. 
\end{homeworkSection}

\begin{homeworkSection}{Francis Phan}
Francis Phan contributed to the ProductManager and Item components, the dummy database, and linking and routing all of the components into one application.
\end{homeworkSection}

\end{homeworkProblem}

%----------------------------------------------------------------------------------------
%	PROBLEM 5
%----------------------------------------------------------------------------------------

\begin{homeworkProblem}[Cut Features]
No features were cut from the project, though some features were pushed back until a server/database is implemented as they are not currently feasible at this time.
\end{homeworkProblem}

\end{document}